\section{基础配置介绍}
该框架能应用于Spring MVC与Struts2中。对于两个不同的框架,配置和使用会有少许不同。
但是,对于表单验证的脚本实现则没有区别。也就是说,不论你是用Spring MVC还是Struts2
验证机制的脚本是通用的,配置也基本类似。所以,两者的配置和实现基本可以互换。

\subsection{验证脚本实现}
Validation验证脚本是用 javascript 实现的。在服务器端可以使用 1.5 标准。
客户端则由具体浏览器决定。使用者需要把验证规则写在一个 javascript 函数中。该函数
接收一个以表单数据名为key,该数据的value所组成的数组的一个map。返回也是一个类似
的map,只是原来用来存放value的数组,变为存放错误信息的数组。对于该脚本的一个特殊
限制是,不可以使用浏览器相关对象以及其成员函数,并且该脚本必须是一个独立的js脚本
文件,不可嵌在页面的script标签中。

以下为一个验证脚本的实例。其中, addErrorField 函数是一个自定义函数便于错误信息的
构建,你可以实现自己的自定义函数。其中 userRegisterValid 就是所要实现的验证脚本
接口了。只要保证输入和输出结构,内部实现机制可以自由发挥。

\begin{verbatim}
function addErrorField(obj, name, msg) {
  if (!obj[name])
    obj[name] = new Array();
  obj[name].push(msg);
}


function userRegisterValid(data) {
  var ret = {};

  if (data["name"][0] == "") {
    addErrorField(ret, "name", "用户名不能为空");

  } else if (data["name"][0].length < 3 || 
             data["name"][0].length > 20) {
    addErrorField(ret,
                  "name",
                  "用户名限制在3到20个字符");
  }

  if (data["password"][0] == "" ||
      data["password"][1] == "") {
    addErrorField(ret,
                  "password",
                  "密码均不能空");
  } else if (data["password"][0] != 
             data["password"][1]) {
    addErrorField(ret,
                  "password",
                  "两个密码必须相同");
  } else if (data["password"][0].length > 8) {
    addErrorField(ret,
                  "password",
                  "密码必须在8个字符以内");
  }
  return ret;
}
\end{verbatim}


\subsection{客户端脚本实现}
Validation 在客户端实现了一个 Jquery 的 plugin 。使用时,只需要调用该plugin即可。
以下为HTML页面内容。核心内容是 formValid 这个 plugin 方法。传入的参数是刚才我们
实现的那个符合接口要求的校验脚本,返回的是一个带有错误信息的map。

\begin{verbatim}
<html>
  <head>
    <title>Hello World</title>
    <script src="jquery.min.js"></script>
    <!-- jquery plugin -->
    <script src="validation.js"></script>
    <script src="validator.js"></script>
    <script>
      $(document).ready(function() {
        $("#formValid").submit(function(e) {
          var err = 
            $("#formValid").formValid(userRegisterValid);
          if (err) {
            var msg = ""
            for (var i in err) {
              for (var j = 0; j < err[i].length; j += 1)
                msg += err[i][j] + "<br/>";
            }
            $("#validResult").html(msg);
            return false;
          }
          return true;
        });
      });
    </script>
  </head>
  <body>
    <div id="validResult" style="color:red"></div>
    <form id="formValid" 
          action="validation.do" 
          method="post">
      <div>
        用户名:<input name="name" type="textfield"/>
      </div>
      <div>
        密码:<input name="password" type="password"/>
      </div>
      <div>
        密码:<input name="password" type="password"/>
      </div>
        <input type="submit"/>
      </div>
    </form>
  </body> 
</html>
\end{verbatim}

