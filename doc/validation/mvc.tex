\section{Spring MVC的配置和使用}
由于MVC的实现机制原因,MVC下的配置和使用略微比struts2复杂一点。其中会涉及一些
MVCSupport的辅助类,用于配置自定义的赋值拦截。MVC中的表单校验结果需要手动处理。
但是,处理起来并不复杂,你可以只加个判断,然后让mvc返回到错误页面上。

\subsection{MVC的配置}
由于MVC是Spring的一部分,所以配置只关联一个文件。一套完整的配置需要建立3个
SpringBean。

以下是spring的配置文件的内容。其中那个HanlerAadaperConfig是本框架用来配置
MVC中的一个Bean的配置的。在Spring3.1中这个有官方方式解决。而valdator的
配置就和Struts2中是一样的。剩下的 validatorHandler 则类似于 Struts2 中的
拦截器的作用。而MethodHanderConfig 则是用来将拦截器绑定到MVC中的Bean。

\begin{verbatim}
<bean name="validator"
      class="org.mustardseed.validation.Validator">
  <property name="scriptPath" value="/script/validator.js"/>
</bean>
  
<bean name="validatorHandler"
  class="org.mustardseed.validation.spring.ValidationHandler">
  <property name="validator" ref="validator"/>
</bean>

<bean class="org.mustardseed.mvc.config.MethodHandlerConfig"/>
\end{verbatim}

\subsection{Validation的使用}

SpringMVC中的使用也是用注解形式标注。不同于 Struts2 把注解标注在 method 上,
MVC是将注解标注在参数上。


以下是MVC中用于表单处理的成员函数。 标注了 FormValidResult 的 Map 将被赋予
校验结果,如果为 null 则说明表单处理没有问题。不然其中包含的就是含有错误信息
的一个Map。
\begin{verbatim}
@RequestMapping(value="/validation",
                method=RequestMethod.POST)
public ModelAndView 
  validation(@RequestParam("name")
               String name,
             @RequestParam("password")
               String password,
             @FormValidResult("userRegisterValid")
               Map<String, String[]> result) {
  ModelAndView mav = new ModelAndView();
  if (result != null) {
    mav.setViewName("valid");
    mav.addObject("error", result);
    mav.addObject("name", name);
    mav.addObject("password", password);
  } else {
    mav.setViewName("redirect:index.do");
  }
  return mav;
}
\end{verbatim}
